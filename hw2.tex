%%%%%%%%%%%%%%%%
% Set options

\newcommand{\settitle}{Homework 1: Many-time pad}
\newcommand{\course}{CSE 207B}
\newcommand{\coursename}{Applied Cryptography}
\newcommand{\assigndate}{March 29, 2022}
\newcommand{\duedate}{Thursday, April 7}

%%%%%%%%%%%%%%%%

\documentclass[letterpaper,12pt]{article}
\usepackage[top=1in, bottom=1in, left=1in, right=1in]{geometry}
\usepackage{fancyvrb}
\usepackage{amsmath,amsthm,amssymb,amsfonts, mathrsfs, mathtools}
\newcommand{\N}{\mathbb{N}}
\newcommand{\Z}{\mathbb{Z}}
\newcommand{\Q}{\mathbb{Q}}
\newcommand{\R}{\mathbb{R}}
\newcommand{\C}{\mathbb{C}}
\newcommand{\F}{\mathbb{F}}
\usepackage[protrusion=true,expansion=auto]{microtype}
\usepackage{color}
\usepackage[T1]{fontenc}
\usepackage{lmodern}
%\usepackage{mathptmx}
\usepackage{textcomp}
\usepackage[
  breaklinks=true,colorlinks=true,linkcolor=black,%
  citecolor=black,urlcolor=black,bookmarks=false,bookmarksopen=false,%
  pdfauthor={\course},%
  pdftitle={\settitle},%
  pdftex]{hyperref}
\usepackage{amsmath}
\usepackage{amsfonts}
\usepackage{multicol}
\renewcommand{\ttdefault}{cmtt}
\def\textsb#1{{\fontseries{sb}\selectfont #1}}

\newcommand{\problemsetdone}{\hfill$\Box{}$}

\newcommand{\htitle}
{
    \vbox to 0.25in{}
    \noindent\parbox{\textwidth}
    {
        \course\hfill \assigndate\newline
        \coursename\hfill
        Due: \duedate \vspace*{-.5ex}\newline
        \mbox{}\hrulefill\mbox{}
    }
    \vspace{8pt}
    \begin{center}{\Large\bf{\settitle}}\end{center}
}
\newcommand{\handout}
{
    \thispagestyle{empty}
    \markboth{}{}
    \pagestyle{plain}
    \htitle
}

\newcommand{\problemsetheader}
{
Your homework should be submitted electronically via Gradescope before class on \duedate.  Please use \LaTeX to format your solutions and submit as \texttt{hw1-solutions.pdf} along with the code you used to solve the first part, which Gradescope expects to be called \texttt{hw1-sol.py}.  Please credit any collaborators you worked with and any sources you used.  You may use the following solution template: \\
\url{https://cseweb.ucsd.edu/classes/sp22/cse207B-a/hw1/hw1-solutions.tex}.

\medskip

You can find the Latex source used to generate this document at \\ \url{https://cseweb.ucsd.edu/classes/sp22/cse207B-a/hw1/hw1.tex}.

\hrulefill

\bigskip
}

%%%%%%%%%%%%%%%%%%%%%%%%%%%%%%%%%%%%%%%%%


%%%%%%%%%%%%%%%%%%%%%%%%%%%%%%%%%%%%%%%%%

\begin{document}
\handout
\setlength{\parindent}{0pt}
\problemsetheader

\begin{enumerate}
\setcounter{enumi}{-1}
\item Describe how you solved the decryption challenge.  Include any code you wrote as a separate file.

\item 

We may also define a ``multiplication mod $p$'' variant of the one-time pad.  This is a cipher $(\operatorname{Enc}, \operatorname{Dec})$, defined over $(\mathcal{K},\mathcal{M}, \mathcal{C})$, where $\mathcal{K} := \mathcal{M} := \mathcal{C} := \{1, \dots, p-1\}$, where $p$ is a prime.  Encryption and decryption are defined as follows:

\[
\operatorname{Enc}_k(m) := k \cdot m \bmod p \qquad \operatorname{Dec}_k(c) := k^{-1} \cdot c \bmod p.
\]
Here, $k^{-1}$ denotes the multiplicative inverse of $k$ modulo $p$.  Verify the correctness property for this cipher and prove that is perfectly secure.

\item

Let $p$ be an odd prime. Prove that for all integers $x$ relatively prime to $p$, $x^{\frac{p-1}{2}} \equiv \pm 1 \pmod{p}$. 
You are free to assume Fermat's little theorem, which says that if $p$ is prime and $x$ is not divisible by $p$, then $x^{p-1} \equiv 1 \pmod{p}$.
You may also use the fact that when working mod a prime, a degree-$d$ polynomial can have no more than $d$ roots.

\item

Two standards committees propose to save bandwidth by combining compression (such as the Lempel-Ziv algorithm used in the zip and gzip programs) with encryption.  Both committees plan on using the variable length one-time pad for encryption.

\begin{itemize}
\item One committee proposes to compress messages before encrypting them.  Explain why this is a bad idea.

\textit{\textbf{Hint:}} Recall that compression can significantly shrink the size of some messages while having little impact on the length of other messages.

\item The other committee proposes to compress ciphertexts after encryption.  Explain why this is a bad idea.

\end{itemize}

\end{enumerate}
\vfill

\end{document}